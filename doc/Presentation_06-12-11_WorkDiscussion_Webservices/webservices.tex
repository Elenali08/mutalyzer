\section{Webservices}
\begin{frame}
  \frametitle{Life without webservices}
  \pause

  Example: Get the first hit in google:
  \begin{itemize}
    \item Figure out what the server expects.
    \item \bt{http://www.google.com/\#q=test}
    \item Parse the resulting HTML file.
  \end{itemize}
  \bigskip
  \pause

  Disadvantages:
  \begin{itemize}
    \item The communication variables can change (\bt{q} changes to
      \bt{query}).
    \item The resulting HTML file can change.
  \end{itemize}
  \bigskip

  Conclusions:
  \begin{itemize}
    \item Requires quite some expertise to set up.
    \item Requires a lot of maintenance.
  \end{itemize}
\end{frame}

\begin{frame}
  \frametitle{SOAP webservices}

  Characteristics of the SOAP webservice:
  \begin{itemize}
    \item Communication XML/RPC over HTTP (not necessarily over port 80).
    \item Description of the interface is machine readable.
  \end{itemize}
  \bigskip

  Communication over HTTP is essential for us (firewall etc.).
  \bigskip
  \pause

  The description of the interface is machine readable:
  \begin{itemize}
    \item The communication protocol can be abstracted.
    \begin{itemize}
      \item The actual communication can change without the client being aware
        of it.
      \item Functions can be added without a need for the client to update.
    \end{itemize}
  \end{itemize}
\end{frame}

\begin{frame}
  \frametitle{SOAP webservices}

  \vspace{-0.5cm}
  \begin{center}
    \resizebox{11cm}{6.5cm}{
      \input{setup.pstex_t}
    }
  \end{center}
\end{frame}

\begin{frame}
  \frametitle{SOAP webservices}
  \bigskip

  \begin{itemize}
    \item The transport/communications layer is completely hidden from both the
          client as well as the server.
    \bigskip
    \item Exported functions are normal local functions on the server side
      (makes testing easy).
    \bigskip
    \item The client sees the functions as local functions (not different from
          functions included from a library).
  \end{itemize}
\end{frame}

\begin{frame}[fragile]
  \frametitle{An example}

  \begin{lstlisting}[caption = {Server side}]
    @soapmethod(String, Integer, _returns = String)
    def sayHello(name, times) :
        return ("Hello " + name + ' ') * times
  \end{lstlisting}

  \begin{lstlisting}[caption = {Client side}]
    from SOAPpy import WSDL
    service = WSDL.Proxy("http://path_to_wsdl.wsdl")
    print service.sayHello("MyName", 10)
  \end{lstlisting}
  \bigskip
  \pause

  \begin{lstlisting}[caption = {Local function (for comparison)}]
    from Bio import pairwise2
    print pairwise2.align("AAAATT", "AATAA")
  \end{lstlisting}
\end{frame}

\begin{frame}[fragile]
  \frametitle{Discovery}

  \begin{itemize}
    \item The client object has a standard function that gives a list of
      function names and a description of the parameters.
    \item The WSDL file also contains full documentation (defined on the
      server).
    \item We also generate documentation from the source code on the website.
    \item Tools for viewing the WSDL are also available.
  \end{itemize}
  \bigskip
  \bigskip
  \pause

  \begin{lstlisting}[caption = {WSDL}]
    from SOAPpy import WSDL
    service = WSDL.Proxy("http://path_to_wsdl.wsdl")
    print service.show_methods()
  \end{lstlisting}
\end{frame}
