\documentclass{article}
\usepackage{fullpage}

\author{J.F.J. Laros \and M. Vermaat \and J.T. den Dunnen \and P.E.M. Taschner}
\title{Disambiguating complex HGVS variant descriptions}

\frenchspacing

\begin{document}

\maketitle

\begin{abstract} \noindent
The \emph{Human Genome Variation Society} (HGVS)~\cite{NOM1} nomenclature for
the description of sequence variations \ldots

\paragraph{Background}
The recent formalisation of the HGVS nomenclature syntax~\cite{hgvs_bnf} makes
it possible to automatically interpret the variant description and reconstruct
the observed sequence. This formalisation however, tells us nothing about how
to make such a description.

\paragraph{Problem description}
Formally, a variant description is, together with the reference sequence, the
input of a function that transforms the reference sequence into the observed
sequence. This function is not injective; multiple descriptions can generate
the same observed sequence. If for example, we observe a change from
\texttt{ATGCTTCAGG} to \texttt{CTGAAGCATT}. The untrained eye might see this
change as \texttt{1\_10delinsCTGAAGCATT}, while the preferred description would
be \texttt{1\_9inv;10G>T}. We call the set of descriptions that result in the
same observed sequence the set of \emph{equivalent descriptions}.

\paragraph{Solution}
We present an algorithm that, given a reference sequence and an observed
sequence, will generate the HGVS description of the variant. Because there is
no direct link between the variant description that is used to reconstruct the
observed sequence and the generated variant description, this algorithm will
always generate the same description, no matter which description in the set of
equivalent descriptions is used.

\paragraph{Implementation}
We start with finding the smallest indel that describes the change by removing
the longest common prefix and the longest common suffix from the reference- and
the observed sequence. Next, we recursively try to find a shorter description
using the following strategy:

First we determine the \emph{longest common substring} (LCS) in both the
forward and the reverse strand. If the LCS is found on the forward strand, we
split the description in two parts and recursively describe the separate parts.
If the LCS is found on the reverse strand, we split the description in three
parts, the same two parts that we would get in the former case, plus an
inversion in between.

The recursion ends if an elementary description (substitution, insertion,
deletion, etc.) is found. If a variant was split, the length of the description
is compared to the length of the indel that was split and the shortest of the
two is returned.

\paragraph{Conclusion}
It works.

\bibliographystyle{plain}
\bibliography{/home/jfjlaros/projects/bibliography.bib}
\end{abstract}

\end{document}
