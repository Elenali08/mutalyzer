\documentclass{article}
\usepackage{fullpage}

\newcommand{\superscript}[1]{\ensuremath{^{\textrm{#1}}}}
\frenchspacing

\author{J.F.J. Laros \and M. Vermaat \and J.T. den Dunnen \and P.E.M. Taschner}
\title{Generating complex descriptions of sequence variants using HGVS
  nomenclature based on sequence comparison.\footnote{Funded in part by the
  European Community's Seventh Framework Programme (FP7/2007-2013) under grant
  agreement n\superscript{o} 200754 - the GEN2PHEN project.}}

\begin{document}

\maketitle

\begin{abstract} \noindent
Descriptions of sequence variants can be checked and corrected with the
\emph{Mutalyzer sequence variation nomenclature
checker}\footnote{\texttt{https://mutalyzer.nl/}} to prevent mistakes and
uncertainties which might contribute to undesired errors in clinical diagnosis.
Construction of variant descriptions accepted by Mutalyzer requires comparison
of the reference sequence and the variant sequence and basic knowledge of the
\emph{Human Genome Variation Society sequence variant nomenclature
recommendations}\footnote{\texttt{http://www.hgvs.org/mutnomen/}}. With the
advert of sophisticated variant callers (e.g., Pindel) and the rise of long
read sequencers (e.g., PacBio), the chance of finding a complex variant
increases and so does the need to describe these variants. An algorithm
performing the sequence comparison would help users to describe complex
variants.

The algorithm closely follows the human approach to describe a variant. It will
first find the ``area of change'', and then finds the largest overlap between
the original area and the area in the observed sequence. This process is
repeated until the smallest description is found.

This algorithm ensures that the same description will be generated every time
researchers observe this variant. Furthermore, no knowledge of the HGVS
nomenclature is required to generate this description. This not only helps
clinicians to generate the correct description, but its implementation also
allows automation of the description process.

We have incorporated this algorithm in the Mutalyzer suite under the name
\emph{Description
Extractor}\footnote{\texttt{https://mutalyzer.nl/descriptionExtract}}.
\end{abstract}

\end{document}
