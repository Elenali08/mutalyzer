\documentclass[slidestop]{beamer}

\title{Mutalyzer 2.0 interfaces}
\providecommand{\myConference}{NBIC conference}
\providecommand{\myDate}{Wednesday, 20 April 2011}
\author{Martijn Vermaat}
\providecommand{\myGroup}{Leiden Genome Technology Center}
\providecommand{\myDepartment}{Department of Human Genetics}
\providecommand{\myCenter}{Center for Human and Clinical Genetics}
\providecommand{\lastCenterLogo}{
  \raisebox{-0.1cm}{
    \includegraphics[scale = 0.1]{gen2phen_logo}
  }
}
\providecommand{\lastRightLogo}{
  \includegraphics[scale = 0.1]{nbic_logo}
}

\usetheme{lumc}

\begin{document}

\providecommand{\positionpicture}{
  \vspace{-0.5cm}
  \begin{center}
    \fbox{
      \begin{picture}(300, 60)(0, 0)
        \put(0, 30){\line(1, 0){300}}  % Genomic sequence.
        \linethickness{4pt}
        \put(50, 30){\line(1, 0){30}}  % Non-coding parts of the exons.
        \put(220, 30){\line(1, 0){10}}
        \linethickness{12pt}
        \put(80, 30){\line(1, 0){20}}  % Coding parts of the exons.
        \put(150, 30){\line(1, 0){20}}
        \put(200, 30){\line(1, 0){20}}

        \linethickness{0.5pt}
        \put(20, 50){\scriptsize{Transcription start}}
        \put(50, 45){\vector(0, -1){10}}
        \put(200, 50){\scriptsize{Transcription end}}
        \put(230, 45){\vector(0, -1){10}}

        \put(70, 0){\scriptsize{CDS start}}
        \put(80, 10){\vector(0, 1){10}}
        \put(210, 0){\scriptsize{CDS stop}}
        \put(220, 10){\vector(0, 1){10}}

        \put(0, 0){\scriptsize{Genomic end}}
        \put(0, 10){\vector(0, 1){10}}
        \put(255, 0){\scriptsize{Genomic start}}
        \put(300, 10){\vector(0, 1){10}}

        \put(95, 50){\color{yellow}\scriptsize{Variant A}\color{white}}
        \put(115, 45){\color{yellow}\vector(0, -1){10}\color{white}}

        \put(140, 50){\color{yellow}\scriptsize{Variant B}\color{white}}
        \put(160, 45){\color{yellow}\vector(0, -1){10}\color{white}}
      \end{picture}
    }
  \end{center}
  \bigskip
}


% This disables the \pause command, handy in the editing phase.
%\renewcommand{\pause}{}

% Make the title page.
\bodytemplate

% First page of the presentation.
\section{Introduction}
\begin{frame}
  A curational tool for \emph{Locus Specific Mutation Databases} (LSDBs).

  \bigskip
  \begin{itemize}
    \pause
    \item Variant nomenclature checker applying \emph{Human Genome Variation
          Society} (HGVS) guidelines.
    \begin{itemize}
      \item Is the syntax of the variant description valid?
      \item Does the reference sequence exist?
      \item Is the variant possible on this reference sequence?
      \item Is this variant description the recommended one?
    \end{itemize}
    \pause
    \item Basic effect prediction.
    \begin{itemize}
      \item Is the description of the transcript product as expected?
      \item Is the predicted protein as expected?
    \end{itemize}
  \end{itemize}
\end{frame}

\section{HGVS nomenclature}
\begin{frame}
  Genomic orientated positions:
  \begin{center}
    \bt{AL449423.14:g.[65449\_65463del;65564\color{yellow}T\color{white}>\color{yellow}C\color{white}]}
  \end{center}
  \pause
  \bigskip
  Coding sequence orientated positions:
  \begin{center}
    \bt{AL449423.14(CDKN2A\_v001):c.[5\color{yellow}A\color{white}>\color{yellow}G\color{white}
      ;106\_120del]}
  \end{center}
  \bigskip
  \pause
  \begin{itemize}
    \item \bt{AL449423.14} -- reference sequence.
    \item \bt{CDKN2A\_v001}$\;$ -- transcript variant \bt{1} of gene CDKN2A.
    \item \bt{c.[5\color{yellow}A\color{white}>\color{yellow}G\color{white};106\_120del]}
    \begin{itemize}
      \item A \emph{substitution} at position \bt{5} counting from the start
        codon.
      \item A \emph{deletion} from position \bt{106} to position \bt{120}.
    \end{itemize}
  \end{itemize}
\end{frame}

\section{HGVS nomenclature: positions}
\begin{frame}
  \positionpicture

  This gene is on the reverse strand.

  \bigskip
  \bigskip
  \pause
  \begin{tabular}{l|l|p{7cm}}
    Name       &         & Description\\
    \hline
    Genomic    & \bt{g.} & From {\scriptsize Genomic start} to
                           {\scriptsize Genomic end}. \\
    Transcript & \bt{n.} & From {\scriptsize Transcription start} to
                           {\scriptsize Transcription end}, skip introns.\\
    Coding     & \bt{c.} & From {\scriptsize CDS start} to
                           {\scriptsize CDS stop}, skip introns.
  \end{tabular}
\end{frame}

\begin{frame}
  \positionpicture

  \bt{c.} positions:
  \bigskip
  \begin{itemize}
    \item Positions in introns are relative to the nearest exonic position.
    \item Positions before the CDS are indicated with a \bt{-} sign.
    \item Positions after the CDS are indicated with a \bt{*} sign.
  \end{itemize}

  \pause
  \bigskip
  Position \bt{-1} and \bt{1} are adjacent.

  If \bt{60} is the last position of the CDS, then \bt{60} and \bt{*1} are
  adjacent.
\end{frame}

\begin{frame}
  \positionpicture

  \renewcommand{\arraystretch}{1}
  \begin{center}
    \begin{tabular}{l|r|r|r}
      Name                              & \bt{g.}  & \bt{n.}      & \bt{c.} \\
      \hline
      {\scriptsize Genomic start}       & \bt{1}   & \bt{100+d70} &
        \bt{*10+d70} \\
      {\scriptsize Genomic end}         & \bt{300} & \bt{1-u50}   &
        \bt{-30-u50} \\
      {\scriptsize Transcription start} & \bt{250} & \bt{1}       & \bt{-30} \\
      {\scriptsize Transcription end}   & \bt{70}  & \bt{100}     & \bt{*10} \\
      {\scriptsize CDS start}           & \bt{220} & \bt{30}      & \bt{1} \\
      {\scriptsize CDS stop}            & \bt{80}  & \bt{90}      & \bt{60} \\
    \end{tabular}
  \end{center}
\end{frame}

\begin{frame}
  \positionpicture

  \renewcommand{\arraystretch}{1}
  \begin{center}
    \begin{tabular}{l|r|r|r}
      Name                    & \bt{g.}  & \bt{n.}    & \bt{c.} \\
      \hline
      {\scriptsize Variant A} & \bt{185} & \bt{50+15} & \bt{20+15} \\
      {\scriptsize Variant B} & \bt{140} & \bt{60}    & \bt{30} \\
    \end{tabular}
  \end{center}

  \bigskip
  \bigskip
  \pause
  \bt{NG\_001234.1:g.185\color{yellow}A\color{white}>\color{yellow}C\color{white}}

  \bt{NG\_001234.1(GEN\_v001):n.50+15\color{yellow}T\color{white}>\color{yellow}G\color{white}}

  \bt{NG\_001234.1(GEN\_v001):c.20+15\color{yellow}T\color{white}>\color{yellow}G\color{white}}
\end{frame}

\section{Mutalyzer 2.0: name checker}
\begin{frame}
  \begin{center}
    \bt{NM\_002001.2:c.[12\_14del;102\color{yellow}G\color{white}>\color{yellow}T\color{white}]}
  \end{center}
  \bigskip
  \begin{enumerate}
    \pause
    \item Parse the variant description.
    \begin{itemize}
      \item Reference sequence e.g., \bt{NM\_002001.2}.
      \item Position system (\bt{c.}, \bt{g.}, \bt{n.}, \ldots).
      \item List of variants (\bt{12\_14del},
            \bt{102\color{yellow}G\color{white}>\color{yellow}T\color{white}}).
    \end{itemize}
    \pause
    \item Download the reference sequence.
    \pause
    \item Check the variants to the reference sequence.
    \begin{itemize}
      \item Is there a \bt{\color{yellow}G\color{white}} at position \bt{c.102}?
    \end{itemize}
    \pause
    \item Mutate the reference sequence.
    \pause
    \item Predict the variant protein when applicable.
    \item \ldots
  \end{enumerate}
\end{frame}

\begin{frame}
  After a description is checked, other useful information is returned.
  \begin{itemize}
    \pause
    \item Overview of the change on DNA level.
    \pause
    \item A genomic description.
    \pause
    \item A description on all affected transcripts.
    \pause
    \item Description of affected proteins.
    \pause
    \item Sequence of the original and affected protein with changes
      highlighted.
    \pause
    \item Exon and CDS start / stop information.
    \pause
    \item Effects on restriction sites.
    \pause
    \item Information about transcripts, prote\"ins, their products and their
      IDs.
  \end{itemize}
\end{frame}

\section{Mutalyzer 2.0: position converter}
\begin{frame}
  Next Generation Sequencing uses chromosomal positions.

  LSDB's usually use transcripts.
  \bigskip

  The position converter:
  \begin{itemize}
    \pause
    \item Works on both hg18 (NCBI Build 36.1) and hg19 (GRCh37).
    \pause
    \item Works in both ways:
    \begin{itemize}
      \item \bt{NM\_003002.2:c.274\color{yellow}G\color{white}>\color{yellow}T\color{white}} to
            \bt{NC\_000011.9:g.111959695\color{yellow}G\color{white}>\color{yellow}T\color{white}}.
      \item \bt{chr11:g.111959695\color{yellow}G\color{white}>\color{yellow}T\color{white}} to
            \bt{NM\_003002.2:c.274\color{yellow}G\color{white}>\color{yellow}T\color{white}}.
    \end{itemize}
    \pause
    \item Can be used to \emph{lift over} from hg18 to hg19 and vice versa.
    \pause
    \item LSDBs use it to map variants to the genome.
    \begin{itemize}
      \item NGS pipelines can find variants.
      \item NGS pipelines can store data in LSDBs.
    \end{itemize}
  \end{itemize}
\end{frame}

\section{Mutalyzer 2.0: other functionalities}
\begin{frame}
  Other functionalities of Mutalyzer 2.0 include:
  \begin{itemize}
    \pause
    \item SNP conversion (from dbSNP rsId to HGVS notation).
    \pause
    \item Syntax checker.
    \pause
    \item Name generator (to help people that don't use the HGVS notation that
          often).
    \pause
    \item GenBank uploader (to make your own reference sequences).
    \begin{itemize}
      \item Automatically uses the correct strand when a HGNC gene symbol is
            used.
    \end{itemize}
    \pause
    \item Recently added functionality for the LRG (Locus Reference Genomic)
          reference files.
    \begin{itemize}
      \item Other formats can easily be added.
    \end{itemize}
  \end{itemize}
\end{frame}

\begin{frame}
  For a large number of checks, there are other interfaces.
  \bigskip
  \begin{itemize}
    \pause
    \item Batch interfaces (upload a table, receive the result by mail):
    \begin{itemize}
      \item Name checker.
      \item Syntax checker.
      \item Position converter.
      \item SNP converter.
    \end{itemize}
    \pause
    \item Programmatic access (use from your own scripts).
    \begin{itemize}
      \item Currently $18$ functions available.
      \begin{itemize}
        \item Position conversion.
        \item Mutate a reference sequence.
        \item Retrieve all transcripts in a range of a chromosome.
        \item Give extensive information of transcripts.
        \item Upload GenBank files.
        \item \ldots
      \end{itemize}
    \end{itemize}
  \end{itemize}
\end{frame}

\section{Questions?}
\lastpagetemplate
\begin{frame}
  \begin{center}
    Acknowledgements:
    \bigskip
    \bigskip

    Jeroen Laros\\
    Gerben Stouten\\
    Gerard Schaafsma\\
    Ivo Fokkema\\
    Jacopo Celli\\
    Peter Taschner\\
    Johan den Dunnen
    \bigskip
    \bigskip
    \bigskip

    \bt{http://www.mutalyzer.nl/}
  \end{center}
\end{frame}

\end{document}
