\newcommand{\thisversion}{$2.0$}
\newcommand{\nomenclatureversion}{$2.0$}

\documentclass{article}
\usepackage{amssymb, amsthm, graphicx, float, textcomp, longtable}
\title{\Huge Mutalyzer \thisversion\\
       Technical Reference Manual}
\author{Jeroen F. J. Laros
        \vspace{10pt}\\
        Department of Human Genetics\\
        Center for Human and Clinical Genetics\\
        \texttt{j.f.j.laros@lumc.nl}}
\date{\today}
\frenchspacing
\setlength{\parindent}{0pt}

\begin{document}

\maketitle
\thispagestyle{empty}
\newpage

\pagenumbering{roman}
\tableofcontents
\newpage

\pagenumbering{arabic}

\section{Introduction} \label{sec:introduction}
This document is intended for developers. It describes the internal workings
of the Mutalyzer \thisversion\ modules and their interactions.

This document is for internal use only.

\newpage

\section{Modules} \label{sec:modules}
Mutalyzer \thisversion\ consists of several modules, which can be used in a
number of combinations. Several possible combinations are described in 
Sections~\ref{sec:programs}~and~\ref{sec:interfaces}.

In this section, we give a global overview of all modules and their function,
for a detailed description of the module interface, please refer to the 
API~\cite{API} documentation.

\subsection{The Config module} \label{subsec:config}
The \texttt{Config} module loads the configuration file
(\texttt{mutalyzer.conf}) and processes the variables. 

\begin{itemize}
\item Numbers will be converted to integers.
\item If necessary, integers will be multiplied by a constant (to convert 
      megabytes to bytes for example).
\item Variables are stored in a specific sub class of the \texttt{Config}
      class. 
\end{itemize}

So all the variables that are associated with the \texttt{Retriever} module for
example, are stored in the \texttt{Config.Retriever} class. Upon making an 
instance of the \texttt{Retriever} class, this object can be passed to it to
initialise the class.

Configuration variables of specific classes are described in their own section.

\subsection{The Output module} \label{subsec:output}
The \texttt{Output} module consists of two main data structures; one for the
storage of \emph{messages} and one for the storage of \emph{output}. Upon
construction of the \texttt{Output} module, a file name should be passed to
the constructor to make the module aware of the program that constructed it.
This is a way to make a distinction between different programs and interfaces
in the log file.

Messages are stored in a list of \emph{message objects}. A message object 
consists of the fields described in Table~\ref{tab:message}.

\begin{table}[H]
\begin{center}
\begin{tabular}{l|l}
Field name & Description \\
\hline
origin      & Name of the module creating this object. \\
level       & Importance of the message (see Table~\ref{tab:messagelevel}). \\
code        & The error code of the message 
              (see Appendix~\ref{subsec:error}). \\
description & A description of the message.
\end{tabular}
\caption{Fields in the message object} \label{tab:message}
\end{center}
\end{table}

In the message object, the \emph{file name} field refers to the program in
which the message was generated, the \emph{severity level}, as described in 
Table~\ref{tab:messagelevel} indicates the class of the message, ranging from a
log message to to fatal error, the \emph{code} field refers to the type of
message and the \emph{description} field is used for a long descriptive
message.

\begin{table}[H]
\begin{center}
\begin{tabular}{l|l|l}
Numeric value & Description & Effect when used as configurable level \\
\hline
$0$           & Debug   & Show all messages. \\
$1$           & Info    & Show all messages except debug messages. \\
$2$           & Warning & Show warning, error and fatal messages. \\
$3$           & Error   & Show error and fatal messages. \\
$4$           & Fatal   & Only show fatal messages. \\
$5$           & Off     & Show nothing. \\
\end{tabular}
\caption{Message levels} \label{tab:messagelevel}
\end{center}
\end{table}

The \emph{code} field can be used by a program that uses this module as a means to alter the control flow, instead of parsing the description (that is likely
to change in later versions). The codes and their meaning can be found in 
Appendix~\ref{subsec:error}.

The \emph{severity level} field is used both for logging as well as for
retrieving messages, both ways are controlled by configuration variables that
are passed to the \texttt{Output} module at the time of construction. 
The \texttt{outputlevel} variable is used by the \texttt{getMessages()} member
function. All messages of which the level exceeds the value of this variable
will be returned as a list. Logging is controlled by the \texttt{loglevel}
variable. All messages of which the level exceeds the values of this variable
will be logged. Additionally, the level $-1$ is reserved for logging
exclusively. Messages that are generated with this level, will not appear when
using the \texttt{getMessages()} function. 

In Table~\ref{tab:outputconfig} all configurable settings of this module are
described.

\begin{table}[H]
\begin{center}
\begin{tabular}{l|l}
Variable    & Description \\
\hline
log         & Name and location of the log file. \\
datestring  & Prefix for each log message. \\
loglevel    & Level at which messages are logged. \\
outputlevel & Level at which messages are returned by \texttt{getMessages()}.
\end{tabular}
\caption{Configuration variables of the \texttt{Output} object} 
\label{tab:outputconfig}
\end{center}
\end{table}

For storage of output, a dictionary is used. The values in this dictionary 
are lists. A pair (key, value) can be set by using the \texttt{addOutput()}
function. Subsequent calls to this function with the same key as parameter,
will expand the list that is pointed to by the key.

For retrieval of output, the \texttt{getOutput()} and
\texttt{getIndexedOutput()} functions can be used. Both functions require a 
key name as argument. The first one returns the entire list pointed at by the
key, while the second one only returns one element of this list. The index of
this element is passed as a parameter. If either the key or (in the latter 
case) the index does not exist, an empty list is returned.

\subsection{The Db module} \label{subsec:db}
The \texttt{Db} module is an interface module for MySQL databases. It consists
of one parent class and several derived classes. The parent class is used
exclusively for its \texttt{query()} function. This function accepts an SQL
query along with a list of parameters for this query. The parameters are
escaped to prevent any SQL injections. After escaping, they are combined with
the query to form the final query that is sent to an open database handle.

The database handle is opened upon construction of the \texttt{Db} object and
is closed when the object instance is destroyed.

Please refer to Appendix~\ref{subsec:dbtables} for a full description of the
SQL tables.

The \texttt{Db} module has the following derived classes:
\begin{itemize}
\item \texttt{Mapping} for the mapping of transcripts and genes.
\item \texttt{Remote} for retrieving updates for the mapping databases.
\item \texttt{Update} for updating the mapping databases.
\item \texttt{Cache} for cache administration.
\item \texttt{Batch} for the batch checker.
\end{itemize}

The \texttt{Mapping} class provides an interface for databases that contain
mapping information. At the time of this writing, mapping information of
human builds \emph{hg18} and \emph{hg19}~\cite{GN} are available. This class 
provides functions to convert chromosomal positions into transcript oriented
positions and vice versa. These conversions are needed for the mapping of 
data in locus specific databases~\cite{LOVD} (from transcript level to
chromosome level), or for the analysis of variants discovered in Next
Generation Sequencing projects (in which case the conversion is likely to be
from chromosome to transcript level).

The \texttt{Remote} class and the \texttt{Update} class provide functions for
the querying of the UCSC databases to retrieve new mapping information and to
merge it into the local database (which is controlled by the \texttt{Mapping}
class).

The \texttt{Cache} class makes use of an internal database (not one of the
mapping databases). This class provides functions that query a table filled
with information about downloaded and uploaded reference sequence files. See
Section~\ref{subsec:retriever} for more information.

The \texttt{Batch} class makes use of the same internal database, it uses a
number of tables that contain information about submitted batch jobs and the
submitters. See Section~\ref{subsec:scheduler} for more information.

The \texttt{Db} class and al its derived classes are configured with the
variables shown in Table~\ref{tab:dbconfig}.

\begin{table}[H]
\begin{center}
\begin{tabular}{l|p{8cm}}
Variable        & Description \\
\hline
internalDb      & Name of the internal database. \\
dbNames         & List of MySQL mapping database names. \\
LocalMySQLuser  & MySQL username for the local databases (internalDb and 
                  dbNames). \\
LocalMySQLhost  & Host name for the local databases. \\
RemoteMySQLuser & MySQL username for the UCSC database. \\
RemoteMySQLhost & Host name for the UCSC database. \\
UpdateInterval  & Retrieve all entries modified within a certain number of 
                  days. \\
TempFile        & Temporary file for updated UCSC mapping information.
\end{tabular}
\caption{Configuration variables of the \texttt{Db} object} 
\label{tab:dbconfig}
\end{center}
\end{table}

\subsection{The Retriever module} \label{subsec:retriever}
The \texttt{Retriever} module consists of one parent class and two derived
classes. The parent class handles all cache administration, after a write to
the cache directory, it checks whether the cache size exceeds the configured
maximum, and if so, deletes the oldest files in the cache until the size is
under the limit again.

It also has a function that calculates an md5sum of a \emph{reference sequence}
file that is to be placed in the cache. This function is especially useful for
one of the derived classes, the \texttt{GenBankRetriever} class, discussed
in Section~\ref{subsubsec:genbankretriever}. Apart from a function that calculates an md5sum, there is also a
function that updates an md5sum of a file. This can happen if the annotation
within a reference sequence file has changed (without changing anything
crucial), so the version number of this retrieved file was not increased. An
other useful function is one that returns a new \texttt{UD} number. \texttt{UD}
numbers are assigned to files that are local to the Mutalyzer installation.
They will also be discussed in Section~\ref{subsubsec:genbankretriever}. 

The \texttt{Retriever} class is configured with variables shown in
Table~\ref{tab:retrieverconfig}.

\begin{table}[H]
\begin{center}
\begin{tabular}{l|p{9cm}}
Variable   & Description \\
\hline
email      & Use this email address for retrieval of records at the NCBI. \\
cache      & The cache directory. \\
cachesize  & The maximum size of the cache in megabytes. \\
maxDldSize & The maximum size of a downloaded reference sequence file in
             megabytes. \\
minDldSize & The minimum size of a downloaded reference sequence file in bytes.
\end{tabular}
\caption{Configuration variables of the \texttt{Retriever} object} 
\label{tab:retrieverconfig}
\end{center}
\end{table}

\subsubsection{The GenBankRetriever class} \label{subsubsec:genbankretriever}
The \texttt{GenBankRetriever} class provides functions to retrieve a
\emph{GenBank} file from the NCBI~\cite{NCBI}. There are several interfaces
that accomplish this task: 

\begin{itemize}
\item Automatic retrieval.
\item Retrieval by slicing a chromosome or contig.
\item Retrieval by gene name.
\item Manual uploading.
\item Uploading by providing a hyperlink.
\end{itemize}

Automatic retrieval works for all \emph{accession numbers} and \texttt{GI}
numbers of GenBank files known to the NCBI. The \texttt{efetch()} function from
the \emph{Biopython}~\cite{BIOPYTHON} package is used to retrieve the record.
After retrieving, the file is parsed with the \texttt{SeqIO.read()} function
(also from Biopython). In this process, the \texttt{GI} number is extracted
from the record. If there were no errors while parsing, the md5sum is
calculated and stored in the internal database together with the accession
number and the \texttt{GI} number in the \texttt{GBInfo} table.

Retrieval by slicing a chromosome or contig is done by providing the
\emph{accession number} and two coordinates of in this reference sequence.
Before retrieving, the anticipated size of the slice is calculated to determine
whether it is not too large (see the \texttt{maxDldSize} configuration
variable). Then, the \texttt{efetch()} function from the Biopython package can
be used to retrieve the sequence of interest. The retrieved slice is placed in
the cache and is given its unique \texttt{UD} number. After retrieval, the
md5sum is calculated and stored in the internal database together with the
accession number of the chromosome or contig, the \texttt{UD} number, the 
orientation and the positions used to make the slice in the \texttt{GBInfo}
table. 

Retrieval by gene name can only be done for builds of genomes that are
considered current by the NCBI. At the time of this writing, for example, a
retrieval of a gene of a human would result in the retrieval of a slice of a
chromosome of human build hg19. This retrieval interface has the official
gene symbol, the species and the size of the upstream and downstream flanking
sequences as parameters. The functions \texttt{esearch()} and
\texttt{esummary()} from the Biopython package are used to find the location
and orientation of the gene. After this, we can follow the same procedure that
is used to retrieve a slice of a contig. If a gene can not be found, but it was
encountered as an alias of one or more official gene symbols, a list of
suggestions is put in the description of a message with code \texttt{ENOGENE}
that is given to the \texttt{Output} object (see Section~\ref{subsec:output}).

Manual uploading is done by providing a stream to the \texttt{uploadrecord()}
member function. This stream is first checked for size (see the
\texttt{maxDldSize} and \texttt{minDldSize} configuration variables) and parsed
to check whether it is a valid GenBank file. If all checks have been passed,
the file is given a \texttt{UD} number and is placed in the cache directory.
The md5sum of the file, as well as its \texttt{UD} number are stored in the
database.

Uploading by providing a hyperlink is done by providing a hyperlink to the
\texttt{downloadrecord()} member function. This function will try to find the
size of the record before actually downloading it. If no information can be
retrieved, or when the size is not within the limits, this function will exit.
After successful retrieval, the file is given a \texttt{UD} number and is
placed in the cache directory. The md5sum of the file, as well as its
\texttt{UD} number and the original hyperlink are stored in the database.

Each interface stores information about the origin and the way of retrieval 
in the local database in the \texttt{GBInfo} table. This is done to ease the
task of retrieving a deleted file from the cache. If for example, an
\texttt{UD} number of a deleted file is given, the database can be queried to
see if there is still enough information to re-retrieve the reference sequence
file. If the way of retrieval was by slicing (and thus also by retrieving via
gene symbol), the coordinates and orientation, together with the accession 
number of the chromosome or contig can be used to make the \texttt{UD}
available again. If necessary, the md5sum is updated.

When a file is deleted where the method of retrieval was via a hyperlink, the
file is downloaded again, but is only revived if and only if the md5sum 
matches. Manually uploaded files can not be revived.

An other use of the md5sum is preventing multiple uploads. If the md5sum is
found in the database, the accession number that is linked to this md5sum will
be used as the accession number. So uploading a known GenBank file for example
can result in the function returning not an \texttt{UD} number, but a normal
accession number.

\subsubsection{The LargeRetriever class} \label{subsubsec:lrgretriever}
% TODO write this section.

\subsection{The Mutator module} \label{subsec:mutator}
The \texttt{Mutator} module provides a class that has a number of standard
mutation functions like \texttt{delM()}, \texttt{insM()}, etc. which are all
wrappers for the private \texttt{\_\_mutate()} member function. This function
operates under the assumption that every mutation can be written as a
\emph{deletion-insertion}. The wrapper functions convert positions to 
\emph{interbase coordinates} when needed and call the \texttt{\_\_mutate()}
function. Each wrapper function stores a triple (descriptive message, old 
sequence, new sequence) in the \texttt{Output} object under the name 
\texttt{visualisation}. The old- and new sequence are returned by the 
\texttt{\_\_mutate()} function.

The \texttt{\_\_mutate()} function performs the following tasks:
\begin{itemize}
\item Visualise each mutation.
\item Report any added or removed \emph{enzyme restriction sites}.
\item Mutate the sequence.
\item Store \emph{shift} information.
\end{itemize}

Visualisation of each mutation goes as follows: The deleted part (if any) in
the old sequence is separated from its flanking sequences, similarly, the
inserted part (if any) in the new sequence is separated from its flanking
sequences. The original sequence and the mutated sequence are trimmed when
necessary. Configuration variables associated with the visualisation are
shown in Table~\ref{tab:mutatorconfig}.

\begin{table}[H]
\begin{center}
\begin{tabular}{l|p{9cm}}
Variable      & Description \\
\hline
flanksize     & Length of the flanking sequences (used in the visualisation of
                mutations). \\
maxvissize    & Maximum length of visualised mutations. \\
flankclipsize & Length of the flanking sequences of the clipped mutations (see
                maxvissize).
\end{tabular}
\caption{Configuration variables of the \texttt{Mutator} object} 
\label{tab:mutatorconfig}
\end{center}
\end{table}

If the original or mutated part of a sequence exceeds the \texttt{maxvissize} 
configuration value, this subsequence is broken up in three parts. The outer
parts consist of the head and tail of the subsequence (of length 
\texttt{flankclipsize}) and the center part of the subsequence is replaced by
a number to indicate the length of the subsequence that is not shown. Now the
shortest subsequence is padded to align the flanking sequences vertically.

Since the visualisation part of the \texttt{\_\_mutate()} function already
selects the flanking sequences, the logical place for doing the enzyme
restriction site analysis is here too. We use the
\texttt{Restriction.Analysis()} function from the Biopython package to get a
dictionary of restriction sites, where the key is the enzyme and the value is a
list of positions. This dictionary is converted to a list where each enzyme can
occur more than once by the \texttt{\_\_makeRestrictionList()} member
functions. To compare both lists, all elements from one list are removed from
the other, yielding either the added or the deleted sites, depending on which
list is subtracted from the other. This comparison is done by the
\texttt{\_\_restrictionDiff()} function.

Since all variations are specified by using the \emph{original} coordinates,
special care must be taken when mutating a sequence. A deletion or insertion
will result in a position shift that will alter the coordinates of all
downstream mutations. Therefore, when a mutation is requested (by calling the
\texttt{\_\_mutate()} function), the coordinates are converted to the correct
ones. This is done by registering each position where a \emph{shift} occurs 
and the length and direction of this shift. By iterating through this list, the
correct coordinate can be calculated (iteration stops when we find a shift
position that is larger than the requested position).

\subsection{The Scheduler module} \label{subsec:scheduler}
% TODO write this section.

\begin{table}[H]
\begin{center}
\begin{tabular}{l|l}
Variable                   & Description \\
\hline
processName                & Name of the batch process. \\
mailFrom                   & Return e-mail address. \\
mailMessage                & Location of the mail template. \\
mailSubject                & Subject of the message. \\
resultsDir                 & Location of the results. \\
PIDfile                    & Location of the PID file. \\
nameCheckOutHeader         & The output header for NameChecking \\
syntaxCheckOutHeader       & The output header for NameChecking \\
positionConverterOutHeader & The output header for NameChecking 
\end{tabular}
\caption{Configuration variables of the \texttt{Scheduler} object} 
\label{tab:schedulerconfig}
\end{center}
\end{table}

\subsection{The File module} \label{subsec:file}
The \texttt{File} module provides a class that can handle different file types,
mainly used for \emph{batch} input files. The class provides functions to
convert \emph{comma separated values} (\emph{CSV} for short), \emph{Microsoft
Excel} and \emph{Open Document Spreadsheet} (\emph{ODS} for short) files. These
files can be converted to one intermediate format (a list of lists) that can be
used by other modules. Configuration variables for this module can be found in
Table~\ref{tab:fileconfig}.

\begin{table}[H]
\begin{center}
\begin{tabular}{l|l}
Variable & Description \\
\hline
bufSize  & Amount of bytes to be read for determining the file type. \\
header   & The obligatory header in batch request files. \\
tempDir  & Directory for temporary files.
\end{tabular}
\caption{Configuration variables of the \texttt{File} object} 
\label{tab:fileconfig}
\end{center}
\end{table}

Note that a CSV file does not need to be comma separated (contrary to what the
name suggests). Therefore, a \emph{sniffer} function is used that tries to
guess the \emph{dialect} of the CSV file. Once the dialect is determined, the
file can be parsed. The sniffer function needs to read part of the file in
order to determine the dialect. The size of this chunk is determined by the
\texttt{bufSize} configuration variable.

To determine which parser should be used, the \texttt{getMimeType} member
function is used. This also uses the \texttt{bufSize} configuration variable to
read part of a stream and subsequently guess its file type. This function
returns two values; the \emph{mime type} and a \emph{description}. The latter
is sometimes needed when a container file format is used (as is the case with
OpenOffice documents). Once the file type is successfully determined, the 
appropriate parser is selected.

\subsection{The GenRecord module} \label{subsec:genrecord}
% TODO write this section.

\begin{table}[H]
\begin{center}
\begin{tabular}{l|p{9cm}}
Variable    & Description \\
\hline
upstream    & Number of upstream nucleotides when searching for a 
              transcript. \\ 
downstream  & Number of downstream nucleotides when searching for a 
              transcript. \\ 
spliceAlarm & \\ 
spliceWarn  &
\end{tabular}
\caption{Configuration variables of the \texttt{GenRecord} object} 
\label{tab:genrecordconfig}
\end{center}
\end{table}

\subsubsection{The GBparser class} \label{subsubsec:gbparser}
% TODO write this section.

\subsubsection{The LRGparser class} \label{subsubsec:lrgparser}
% TODO write this section.

\subsection{The Parser module} \label{subsec:parser}
The \texttt{Parser} module provides the \texttt{Nomenclatureparser} class. This
class has only one public function, the \texttt{parse()} function.

The class almost solely consist of \emph{context free grammar} rules in
\emph{BNF} format. This collection of rules define the current version of the
\emph{Human Genome Variation Society} or \emph{HGVS}~\cite{HGVS}
nomenclature~\cite{NOM1, NOM2, NOM3, NOM4, NOM5, NOM6, NOM7}. The version of
the nomenclature is \nomenclatureversion\ at the time of this writing.

A preliminary set of BNF rules is defined in~\cite{BNF}, we extended this set
to include the complete HGVS nomenclature, extended with things like
\emph{chimerism} and \emph{mosaicism}. Also the \emph{nesting} of variants is
supported. This makes the grammar we use far more advanced than the official
HGVS nomenclature. Please see Appendix~\ref{subsec:bnf} for a full description
of the implemented rules.

The \emph{Pyparsing}~\cite{PYPARSING} package is used to write a context free
grammar as Python code. The code practically looks the same as BNF. By adding a
name to a rule (or part of a rule), a \emph{parse tree} is built. This tree is
returned as the result as the parsing process.

If parsing fails, an error message is generated with the \texttt{EPARSE} code
and an output object named \texttt{parseError} is created with a visualisation
of where parsing failed to continue.

\subsection{The Misc module} \label{subsec:misc}
The \texttt{Misc} module provides a class \texttt{Misc} that contains various
functions that are not related to anything else. Presently, it only contains
a function that returns a unique ID used for \texttt{UD} numbers and results
from batch jobs.

\subsection{The Crossmap module} \label{subsec:crossmap}
The \texttt{Crossmap} module provides a class that contains functions to
convert from a \emph{genomic coordinate} to a \emph{transcript oriented}
coordinate or a \emph{coding sequence oriented} coordinate. In the remainder
of this section we shall refer to genomic coordinates as \emph{g.}, transcript oriented as \emph{n.} and coding sequence oriented as \emph{c.} coordinates.
Coordinate conversion is done by linking each splice site to either an n. or
g. coordinate. 

The class is initialised by passing an RNA list, a CDS list and an orientation
of the transcript to the constructor. If a CDS is given, the resulting
crossmapper will convert from g. to c. coordinates. If no CDS is given, it will
convert g. to n. coordinates. 

Most internal functions are operators that deal with mathematical exceptions
when using the ``biological'' coordinates. The number $0$ does not exist for
example, making $-1$ and $1$ neighbouring integers, which makes calculations
awkward. These internal functions take care of these problems.

When a crossmapper is made that converts from g. to c. (a CDS was given at 
construction time), the \emph{CDS stop} coordinate in c. notation is stored 
separately. Internally the crossmapper does not work with ``star'' (*)
positions for positions between CDS stop and \emph{transcription stop}. The
conversion to ``star''-notation is done at the very end.

After initialisation, two main functions are available, \texttt{g2x} and 
\texttt{x2g}. Their behaviour depends upon the initialisation of the class and
is described in Table~\ref{tab:behaviour}.

\begin{table}[H]
\begin{center}
\begin{tabular}{l|l|l}
Initialisation & Function       & Behaviour \\
\hline
RNA only       & \texttt{g2x()} & g. to n. \\
RNA only       & \texttt{x2g()} & n. to g. \\
RNA plus CDS   & \texttt{g2x()} & g. to c. \\
RNA plus CDS   & \texttt{x2g()} & c. to g.
\end{tabular}
\caption{Behaviour of the conversion functions.} \label{tab:behaviour}
\end{center}
\end{table}

% TODO include figure from Peter.

The function \texttt{g2x()} takes a positive integer as only argument and
returns a tuple $(a, b)$, where $a$ is the ``main'' coordinate in c. or n.
notation, this coordinate refers to any position that is within the transcript.
The second element of the tuple $b$ is the ``offset'' coordinate, this
coordinate refers to the distance to the closest \emph{splice site}. A tuple
$(21, -2)$ for example, means that there is a splice site at position $21$ and
that the coordinate of interest lies $2$ positions before this splice site (in
an \emph{intron}). 

As mentioned before, this function does not take CDS stop into account. It is 
not concerned with positions upstream or downstream of the transcript either. 
These final conversions are discussed further on in this section.

The function \texttt{x2g()} takes a tuple $(a, b)$ as argument (discussed
above) and returns a positive integer.

The most important wrapper functions of the \texttt{Crossmap} class are:
\begin{itemize}
\item \texttt{int2main()} converts a $a$ coordinate to a string in
      ``star''-notation.
\item \texttt{main2int()} converts a string in ``star''-notation to the $a$ 
      coordinate.
\item \texttt{int2offset()} converts the $b$ coordinate to an
      ``offset''-notation, adding an ``u'' for offsets upstream of the
      transcript and a ``d'' for offsets downstream of the transcript.
\item \texttt{offset2int()} removes any ``u'' or ``d'' from the offset and
      converts the remainder to an integer.
\end{itemize}

Other wrapper functions in this class only call the wrappers described above.

Finally, there are a couple of functions that offer more information of a 
transcript; an \texttt{info()} function that gives information about
transcription start, transcription stop and CDS stop, a function
\texttt{getSpliceSite()} that returns the genomic coordinate of a splice site
(indexed by rank), and functions that return the number of introns and exons.

\newpage

\section{Programs} \label{sec:programs}
In this section we discuss a number of possible interconnections between the
modules.

\subsection{Mutalyzer} \label{subsec:mutalyzer}
The \texttt{Mutalyzer} program (also known as the \texttt{name checker} perform
several tasks; \emph{semantic checking}, \emph{position conversion},
\emph{visualisation} and \emph{effect prediction} to some extent. 
%TODO more

\subsubsection{Semantic checks}
The main function of \texttt{Mutalyzer} program, is \emph{semantic name
checking}. It uses the \texttt{Parser} module for syntactic checking and the
generation of a parse tree, and applies the given description to a reference 
sequence. This reference sequence is provided by either the \texttt{LRGparser}
of \texttt{GBparser} module. By combining the variant description and the
interpreted reference sequence file, semantic checking can be done.

Semantic checks focus on several aspects of a variant description:
\begin{itemize}
\item Correctness of an intronic position notation.
\item Disambiguation of a deletion or insertion.
\item Minimising a description.
\item Verifying optional arguments.
\end{itemize}

As an example of an intronic position check, we look at the position $25+4$. In
this case, position $25$ must be a splice donor site.

When a nucleotide is deleted from a stretch of nucleotides, the HGVS prescribes
that the last one should be deleted. This disambiguation also applies for more
complex structures; deleting \verb#GACTATTCCCTG# from a sequence
\verb#ATGGACTATTCCCTGGCT# for example, should be corrected to the deletion of
\verb#ACTATTCCCTGG#. Only deletions, insertions and duplications are sensitive
to this kind of ambiguity.

Every description can be written as a deletion/insertion (\emph{delins})
operation, but using this representation is not necessarily the shortest one. A
delins of one nucleotide, for example should be written as a substitution.
Several checks are implemented to reduce one representation to an other. In 
Table~\ref{tab:erosion} a list of possibly ``disguised'' variants is shown.

\begin{table}[H]
\begin{center}
\begin{tabular}{l|l}
Variant type & Could be \\
\hline
delins & del, ins, dup, subst, inv \\
ins    & dup \\
inv    & subst 
\end{tabular}
\end{center}
\caption{Disguised variants.} \label{tab:erosion}
\end{table}

Apart from this possible disguise, a variant can be described using too much
nucleotides, or a range that is too large. A delins can include part of the
flanking sequence of the real change. To prevent this, the \emph{longest common
prefix} and \emph{longest common suffix} are pruned. Likewise, an inversion 
can be too large when the \emph{reverse complement} of a prefix is a suffix.
This is pruned in a similar way. Finally, a delins, subst or inv can have no
effect at all.

Any optional arguments must also be checked. If a length of a deletion is given
for example, it must match the range of the deletion. If a stretch of
nucleotides is given, this stretch must match the reference sequence.

\subsubsection{Effect prediction} \label{subsubsec:effect}
The effect prediction is very limited. \texttt{Mutalyzer} is able to detect
variants that hit a splice site and a start codon and issue a warning
accordingly. From the generated variant description however, some effects can
be deduced. A variant that is at position $x+2$ for example, has a high 
probability of destroying a splice site.

In case of a variant in the start codon, a warning is given, but in case the
variant still has a valid start codon, the analysis continues.

The most informative effect prediction is the \emph{protein description}. This
is made by comparing the original protein and the mutated protein. If only one
raw variant is present, a protein description is made.

There are two major cases to distinguish:
\begin{itemize}
\item A variant that disrupts the reading frame.
\item A variant that does not disrupt the reading frame.
\end{itemize}

To distinguish between these cases, a combination of the \texttt{CrossMap}
and \texttt{Mutator} functions is used. 

\begin{itemize} % FIXME Is this true? Not just CDS len?
\item Take the original CDS stop variable, located in the \texttt{CrossMap} 
      object.
\item Convert this to a genomic position (also using the \texttt{CrossMap}
      module.
\item Use the \texttt{Mutator} module to get the location of the CDS stop 
      variable in the new (mutated) coordinate system.
\item Convert this new position back to a CDS oriented position.
\end{itemize}

If we now look whether this new position is dividable by three, the reading
frame is not disrupted, otherwise it is.

In case of a disrupted reading frame, we first calculate the \emph{longest
common prefix}, and use the remainder of the mutated protein to generate
a \emph{frame shift} notation. This is handled by the \texttt{findFrameShift()}
function.

If the reading frame is not disrupted, we first calculate the \emph{longest
common prefix}, remove this from both the original and mutated protein and then
by calculating and removing the \emph{longest common suffix}. The string that
is left, is in general a delins, but can often be simplified by using a number
of rules. These rules are applied in the \texttt{findInFrameDescription()} 
function.

Also, effects on restriction sites are reported, as described in 
Section~\ref{subsec:mutator}

Both of these functions are wrapped by the \texttt{\_\_toProtDescr()} function
that decides which one to use as described above.

Apart from returning a description, the position where both proteins start
to differ (the \emph{lcp}) and the positions from which both proteins are equal
again (the length of each sequence minus the \emph{lcs}) are returned. This is
used later on for visualisation.

\subsubsection{Visualisation} \label{subsubsec:visualisation}
To aid in the analysis of both variants on DNA as well as the derived protein
description, a number of visualisations are made. First of all, all raw
variants are shown as described in Section~\ref{subsec:mutator}. 

Apart from this, both the original and the mutated protein are shown and the
differences between them are highlighted. This is done by using the positions
returned by the \texttt{\_\_toProtDescr()} function as described in 
Section~\ref{subsubsec:effect}.

\subsection{VarInfo} \label{subsec:varinfo}
The VarInfo interface is a (soon to be deprecated) interface for
LOVD~\cite{LOVDD}. It uses accumulated data from the UCSC~\cite{UCSC} database
to map variants in transcripts to a chromosome and vice versa. At the moment of
this writing, only human builds hg18 and hg19 are supported. Updating of this
local database is describe in Section~\ref{subsec:ucsc_update} and the
structure is described in Appendix~\ref{subsec:dbtables}.

This interface can either be called from the command line or via the web
interface (see Section~\ref{subsec:webinterface}). The interface can be used
in two ways; to get general information about a transcript, and to map a
variant.

By supplying the human build, an accession number and a variant, the interface
will use the local database to generate the values shown in
Table~\ref{tab:varinfomap}.

\begin{table}[H]
\begin{tabular}{l|p{9cm}}
name                   & description\\
\hline
\texttt{start\_main}   & The ``main'' coordinate in c. notation of the start
                         position.\\
\texttt{start\_offset} & The ``offset'' coordinate in c. notation of the start
                         position.\\
\texttt{end\_main}     & The ``main'' coordinate in c. notation of the end
                         position.\\
\texttt{end\_offset}   & The ``offset'' coordinate in c. notation of the end
                         position.\\
\texttt{start\_g}      & The g. coordinate of the start position.\\
\texttt{end\_g}        & The g. coordinate of the end position.\\
\texttt{type}          & The mutation type.\\
\end{tabular}
\caption{Mapping output of the VarInfo interface} \label{tab:varinfomap}
\end{table}

If we map a variant \texttt{NM\_002001.2:c.55+30del}, the output will be:
% 55
% 30
% 55
% 35
% 159272239
% 159272244
% del

\subsection{UCSC\_Update} \label{subsec:ucsc_update}

\newpage

\section{Interfaces} \label{sec:interfaces}
% handler.py

\subsection{Web} \label{subsec:webinterface}
% index.py

The web interface is implemented as a \texttt{WSGI} using the
\texttt{web.py}~\cite{WEBPY} framework. It can be accessed through Apache
using the \texttt{mod\_wsgi} module, or through the built-in webserver by
running \texttt{src/wsgi.py} from the command line.

\emph{Template Attribute Language}~\cite{TAL} (\emph{TAL} for short) is used
to compile template files to HTML files. % FIXME describe menu.

\subsection{Webservices} \label{subsec:webservinterface}
% webservice.py

Much of the Mutalyzer functionality is also available trough a \texttt{SOAP}
webservice, implemented using the Python \texttt{soaplib} library. This
webservice has a separate entrypoint, different from the web interface, which
can be used through Apache/\texttt{mod\_wsgi}, or through its built-in
webserver by running \texttt{src/webservice.py} from the command line.

Client programs can read the \texttt{WSDL} description of the webservice by
a \texttt{GET} request with a \texttt{?wsdl} querystring (e.g.
\texttt{http://mutalyzer.nl/2.0/service?wsdl}).

\subsection{Command line} \label{subsec:commandline}

\newpage

\section{Auxiliary files}

\subsection{mutalyzer.conf}

\subsection{mutalyzer.log}

\subsection{The cache directory}

\subsection{templates}

\newpage

\appendix

\section{Appendix}

\subsection{Error codes} \label{subsec:error}
\begin{table}[H]
\begin{center}
\begin{tabular}{l|l|l}
Code & Class & Description \\
\hline
INFO & N & Information.
\end{tabular}
\end{center}
\caption{Informative messages} \label{tab:info}
\end{table}

\begin{table}[H]
\begin{center}
\begin{tabular}{l|l|p{7cm}}
Code & Class & Description \\
\hline
WSTART      & E & Mutation in the start codon. \\
WTXSTART    & E & Mutation hits transcription start. \\
WSPLDON     & E & Mutation hits a splice donor site. \\
WSPLACC     & E & Mutation hits a splice acceptor site. \\
WROLL       & D & Variant position is ambiguous and not the last one was 
                  given. \\
WINSDUP     & D & Variant was described as an insertion, but it is a
                  duplication. \\
WNOMRNA     & R & No mRNA field was found in the GenBank record. \\
WNOCDS      & R & No CDS field was found in the GenBank record. \\
WNOCDSLIST  & R & No CDS list was found in the GenBank record. \\
WNOVER      & R & No accession version number was given. \\
WHASH       & N & Hash of a GenBank record has changed. \\
WNOCHANGE   & D & Variant equals reference sequence. \\
WNOTMINIMAL & D & A shorter description of a raw variant is possible. \\
WPEND       & R & The LRG file is still pending \\
WDIFFFOUND  & D & 
\end{tabular}
\end{center}
\caption{Warnings} \label{tab:warnings}
\end{table}

\begin{table}[H]
\begin{center}
\begin{tabular}{l|l|p{7cm}}
Code & Class & Description \\
\hline
ENOVAR      & D & No mutation given. \\
EARGLEN     & D & There was a discrepancy between the range and the length of
                  the optional argument. \\
EREF        & D & There was a discrepancy between the reference sequence and 
                  the optional argument. \\
EINSRANGE   & D & The positions of an insertion are not consecutive. \\
WNOCDS      & R & No CDS field was found in the GenBank record and none could
                  be constructed. \\
ENOGENE     & R & Gene not found. \\
ESTOP       & R & In frame stop codon found. \\
EPOS        & D & Invalid position.
\end{tabular}
\end{center}
\caption{Errors} \label{tab:errors}
\end{table}

\begin{table}[H]
\begin{center}
\begin{tabular}{l|l|p{6cm}}
Code & Class & Description \\
\hline
EPARSE             & D & Nomenclature parse error. \\
EBPARSE            & D & Parse error in the submitted batch file. \\
ERECPARSE          & R & GenBank record parse error. \\
EFILESIZE          & N & The filesize is either too large or too small. \\
ERETR              & R & Could not retrieve a GenBank record. \\
EARG               & N & Error in the arguments (of a webservice). \\
ERANGE             & D & Position out of range (webservice). \\
EONLYGB            & D & Conversion only supports GenBank Files \\
EACCNOTINDB        & E & Accession Number Not in Database \\
ENOVERSION         & D & Must supply a version number \\
EINVALIDGENE       & D & Invalid gene name given. \\
EINVALIDTRANSCRIPT & D & Invalid gene name given.
\end{tabular}
\end{center}
\caption{Fatal errors} \label{tab:fatal}
\end{table}

\begin{table}[H]
\begin{center}
\begin{tabular}{l|l}
Class & Description \\
\hline
N & No Error \\
E & Unspecified \\
R & Record related \\
D & Description error
\end{tabular}
\end{center}
\caption{Legend} \label{tab:legend}
\end{table}


\subsection{Database definitions} \label{subsec:dbtables}
\subsubsection{Creating the user}

\begin{verbatim}
CREATE USER mutalyzer;
\end{verbatim}

\subsubsection{Creating the mapping databases}
\paragraph{Human genome build 18}

The table definitions of \texttt{refLink}, \texttt{refGene} and
\texttt{gbStatus} and their content can be found at \\
\verb#http://hgdownload.cse.ucsc.edu/goldenPath/hg18/database/#.

\vbox{
\begin{verbatim}
CREATE DATABASE hg18;
GRANT ALL PRIVILEGES ON hg18.* TO mutalyzer;
FLUSH PRIVILEGES;
\end{verbatim}
}

\vbox{
\begin{verbatim}
USE hg18;
CREATE TABLE map
  SELECT DISTINCT acc, version, txStart, txEnd, cdsStart, 
                  cdsEnd, exonStarts, exonEnds, name2 
                  AS geneName, chrom, strand, protAcc
    FROM gbStatus, refGene, refLink
    WHERE type = "mRNA"
    AND refGene.name = acc
    AND acc = mrnaAcc;
\end{verbatim}
}

\vbox{
\begin{verbatim}
CREATE TABLE map_cdsBackup LIKE map;
\end{verbatim}
}

\vbox{
\begin{verbatim}
CREATE TABLE ChrName (
  AccNo CHAR(20) PRIMARY KEY,
  name CHAR(20) NOT NULL
);
\end{verbatim}
}

\vbox{
\begin{verbatim}
INSERT INTO ChrName (AccNo, name) VALUES
  ("NC_000001.9", "chr1"),
  ("NC_000002.10", "chr2"),
  ("NC_000003.10", "chr3"),
  ("NC_000004.10", "chr4"),
  ("NC_000005.8", "chr5"),
  ("NC_000006.10", "chr6"),
  ("NC_000007.12", "chr7"),
  ("NC_000008.9", "chr8"),
  ("NC_000009.10", "chr9"),
  ("NC_000010.9", "chr10"),
  ("NC_000011.8", "chr11"),
  ("NC_000012.10", "chr12"),
  ("NC_000013.9", "chr13"),
  ("NC_000014.7", "chr14"),
  ("NC_000015.8", "chr15"),
  ("NC_000016.8", "chr16"),
  ("NC_000017.9", "chr17"),
  ("NC_000018.8", "chr18"),
  ("NC_000019.8", "chr19"),
  ("NC_000020.9", "chr20"),
  ("NC_000021.7", "chr21"),
  ("NC_000022.9", "chr22"),
  ("NC_000023.9", "chrX"),
  ("NC_000024.8", "chrY"),
  ("NC_001807.4", "chrM"),
  ("NT_113891.1", "chr6_cox_hap1"),
  ("NT_113959.1", "chr22_h2_hap1");
\end{verbatim}
}

\paragraph{Human genome build 19}
The table definitions of \texttt{refLink}, \texttt{refGene} and
\texttt{gbStatus} and their content can be found at \\
\verb#http://hgdownload.cse.ucsc.edu/goldenPath/hg19/database/#.

\vbox{
\begin{verbatim}
CREATE DATABASE hg19;
GRANT ALL PRIVILEGES ON hg19.* TO mutalyzer;
FLUSH PRIVILEGES;
\end{verbatim}
}

\vbox{
\begin{verbatim}
USE hg19;
CREATE TABLE map
  SELECT DISTINCT acc, version, txStart, txEnd, cdsStart, 
                  cdsEnd, exonStarts, exonEnds, name2 
                  AS geneName, chrom, strand, protAcc
    FROM gbStatus, refGene, refLink
    WHERE type = "mRNA"
    AND refGene.name = acc
    AND acc = mrnaAcc;
\end{verbatim}
}

\vbox{
\begin{verbatim}
CREATE TABLE map_cdsBackup LIKE map;
\end{verbatim}
}

\vbox{
\begin{verbatim}
CREATE TABLE ChrName (
  AccNo CHAR(20) PRIMARY KEY,
  name CHAR(20) NOT NULL
);
\end{verbatim}
}

\vbox{
\begin{verbatim}
INSERT INTO ChrName (AccNo, name) VALUES
  ("NC_000001.10", "chr1"),
  ("NC_000002.11", "chr2"), 
  ("NC_000003.11", "chr3"), 
  ("NC_000004.11", "chr4"), 
  ("NC_000005.9", "chr5"), 
  ("NC_000006.11", "chr6"), 
  ("NC_000007.13", "chr7"), 
  ("NC_000008.10", "chr8"), 
  ("NC_000009.11", "chr9"), 
  ("NC_000010.10", "chr10"), 
  ("NC_000011.9", "chr11"), 
  ("NC_000012.11", "chr12"), 
  ("NC_000013.10", "chr13"), 
  ("NC_000014.8", "chr14"), 
  ("NC_000015.9", "chr15"), 
  ("NC_000016.9", "chr16"), 
  ("NC_000017.10", "chr17"), 
  ("NC_000018.9", "chr18"), 
  ("NC_000019.9", "chr19"), 
  ("NC_000020.10", "chr20"), 
  ("NC_000021.8", "chr21"), 
  ("NC_000022.10", "chr22"), 
  ("NC_000023.10", "chrX"), 
  ("NC_000024.9", "chrY"), 
  ("NT_167244.1", "chr6_apd_hap1"), 
  ("NT_113891.2", "chr6_cox_hap2"), 
  ("NT_167245.1", "chr6_dbb_hap3"), 
  ("NT_167246.1", "chr6_mann_hap4"), 
  ("NT_167247.1", "chr6_mcf_hap5"), 
  ("NT_167248.1", "chr6_qbl_hap6"), 
  ("NT_167249.1", "chr6_ssto_hap7"), 
  ("NT_167250.1", "chr4_ctg9_hap1"), 
  ("NT_167251.1", "chr17_ctg5_hap1")
;
\end{verbatim}
}

\subsubsection{The internal database}

\vbox{
\begin{verbatim}
CREATE DATABASE mutalyzer;
GRANT ALL PRIVILEGES ON mutalyzer.* TO mutalyzer;
FLUSH PRIVILEGES;
\end{verbatim}
}

\vbox{
\begin{verbatim}
USE mutalyzer;
CREATE TABLE GBInfo (
  AccNo CHAR(20) PRIMARY KEY,
  GI CHAR(13) UNIQUE,
  hash CHAR(32) UNIQUE NOT NULL,
  ChrAccVer CHAR(20),
  ChrStart INT(12),
  ChrStop INT(12),
  orientation INT(2),
  url CHAR(255)
);
\end{verbatim}
}

\vbox{
\begin{verbatim}
CREATE TABLE BatchQueue (
  QueueID INT(5) PRIMARY KEY AUTO_INCREMENT,
  JobID CHAR(20) NOT NULL,
  Input CHAR(255) NOT NULL,
  Flags CHAR(20)
);
\end{verbatim}
}

\vbox{
\begin{verbatim}
CREATE TABLE BatchJob (
  JobID CHAR(20) PRIMARY KEY,
  Filter CHAR(20) NOT NULL,
  EMail CHAR(255) NOT NULL,
  FromHost Char(255) NOT NULL,
  JobType CHAR(20) NULL,
  Arg1 char(20) DEFAULT NULL
);
\end{verbatim}
}

\vbox{
\begin{verbatim}
CREATE TABLE Link (
  mrnaAcc CHAR(20) PRIMARY KEY,
  protAcc CHAR(20) UNIQUE NOT NULL
);
\end{verbatim}
}

\vbox{
\begin{verbatim}
INSERT INTO Link 
  SELECT DISTINCT acc, protAcc 
    FROM hg18.map 
    WHERE protAcc != '' 
  UNION 
  SELECT DISTINCT acc, protAcc 
    FROM hg19.map 
    WHERE protAcc != '';
\end{verbatim}
}


\subsection{HGVS nomenclature BNF} \label{subsec:bnf}
\begin{longtable}{llp{7cm}}
name     & $\rightarrow$ & ([a--z][A--Z][0--9])$^+$\\
Nt       & $\rightarrow$ & `a' $|$ `c' $|$ `g' $|$ `t' $|$ `u' $|$ `A' $|$ 
                           `C' $|$ `G' $|$ `T' $|$ `U'\\
NtString & $\rightarrow$ & (Nt)$^+$\\
Number & $\rightarrow$ & ([0--9])$^+$\\
TransVar   & $\rightarrow$ & `\_v' Number\\
ProtIso    & $\rightarrow$ & `\_i' Number\\
GeneSymbol & $\rightarrow$ & `(' name (TransVar $|$ ProtIso)? `)'\\
GI         & $\rightarrow$ & (`GI' $|$ `GI:')? Number\\
Version    & $\rightarrow$ & `.' Number\\
AccNo      & $\rightarrow$ & `ish.'? ([a--Z] Number `\_')$^+$ Version? \\
UD         & $\rightarrow$ & `UD\_' name (`\_' Number)$^+$\\
LRGTranscriptID & $\rightarrow$ & `t' ([0-9])$^+$\\
LRGProteinID    & $\rightarrow$ & `p' ([0-9])$^+$\\
LRG             & $\rightarrow$ & `LRG' ([0-9])$^+$ (`\_' (LRGTranscriptID $|$ 
                  LRGProteinID))?\\
GenBankRef  & $\rightarrow$ & (GI $|$ AccNo $|$ UD) (`(' GeneSymbol `)')?\\
RefSeqAcc & $\rightarrow$ & GenBankRef $|$ LRG\\
Chrom & $\rightarrow$ & name\\
Offset & $\rightarrow$ & (`+' $|$ `-') (`u' $|$ `d')? (Number $|$ `?')\\
RealPtLoc & $\rightarrow$ & ((`-' $|$ `*')? Number Offset?) $|$ `?'\\
IVSLoc & $\rightarrow$ & `IVS' Number (`+' $|$ `-') Number\\
PtLoc  & $\rightarrow$ & IVSLoc $|$ RealPtLoc\\
RefType & $\rightarrow$ & (`c' $|$ `g' $|$ `m' $|$ `n' $|$ `r') `.'\\
Ref & $\rightarrow$ & ((RefSeqAcc $|$ GeneSymbol) `:')? RefType?\\
RealExtent & $\rightarrow$ & PtLoc `\_' (`o'? (RefSeqAcc $|$ GeneSymbol) `:')? PtLoc\\
EXLoc & $\rightarrow$ & `EX' Number (`-' Number)?\\
Extent & $\rightarrow$ & RealExtent $|$ EXLoc\\
RangeLoc & $\rightarrow$ & Extent $|$ `(` Extent `)'\\
Loc & $\rightarrow$ & PtLoc $|$ RangeLoc\\
FarLoc & $\rightarrow$ & (RefSeqAcc $|$ GeneSymbol) (`:' RefType? Extent)? \\
Subst & $\rightarrow$ & PtLoc Nt `$>$' Nt\\
Del & $\rightarrow$ & Loc `del' (Nt$^+$ $|$ Number)?\\
Dup & $\rightarrow$ & Loc `dup' (Nt$^+$ $|$ Number)? Nest?\\
AbrSSR & $\rightarrow$ & PtLoc  Nt$^+$ `(' Number `\_' Number `)'\\
VarSSR & $\rightarrow$ & (PtLoc  Nt$^+$ `[' Number `]') $|$ (RangeLoc `[' Number `]') \\
Ins & $\rightarrow$ & RangeLoc `ins' (Nt$^+$ $|$ Number $|$ RangeLoc $|$ FarLoc) Nest?\\
Indel & $\rightarrow$ & RangeLoc `del' (Nt$^+$ $|$ Number)? `ins' (Nt$^+$ $|$ Number $|$ RangeLoc $|$ FarLoc) Nest?\\
Inv & $\rightarrow$ & RangeLoc `inv' (Nt$^+$ $|$ Number)? Nest?\\
Conv & $\rightarrow$ & RangeLoc `con' FarLoc Nest?\\
ChromBand & $\rightarrow$ & (`p' $|$ `q') Number `.' Number\\
ChromCoords & $\rightarrow$ & `(' Chrom `;' Chrom `)' `(' ChromBand `;' ChromBand `)'\\
TransLoc & $\rightarrow$ & `t' ChromCoords `(' FarLoc `)'\\
RawVar & $\rightarrow$ & Subst $|$ Del $|$ Dup $|$ VarSSR $|$ Ins $|$ Indel $|$ Inv $|$ Conv\\
SingleVar & $\rightarrow$ & Ref RawVar $|$ TransLoc\\
ExtendedRawVar & $\rightarrow$ & RawVar $|$ `=' $|$ `?'\\
CAlleleVarSet & $\rightarrow$ & ExtendedRawVar (`;' ExtendedRawVar)$^*$\\
UAlleleVarSet & $\rightarrow$ & (CAlleleVarSet $|$ (`(' CAlleleVarSet `)')) `?'?\\
SimpleAlleleVarSet & $\rightarrow$ & (`[' UAlleleVarSet `]') $|$ ExtendedRawVar\\
MosaicSet & $\rightarrow$ & (`[' SimpleAlleleVarSet (`/' SimpleAlleleVarSet)$^*$ `]') $|$ SimpleAlleleVarSet\\
ChimeronSet & $\rightarrow$ & (`[' MosaicSet (`//' MosaicSet)$^*$ `]') $|$ MosaicSet\\
SingleAlleleVarSet & $\rightarrow$ & (`[' ChimeronSet ((`;' $|$ `\verb#^#') ChimeronSet)$^*$ (`(;)' ChimeronSet)$^*$ `]') $|$ ChimeronSet\\
SingleAlleleVars & $\rightarrow$ & Ref SingleAlleleVarSet\\
MultiAlleleVars & $\rightarrow$ & Ref SingleAlleleVarSet (`;' Ref? SingleAlleleVarSet)$^+$\\
MultiVar & $\rightarrow$ & SingleAlleleVars $|$ MultiAlleleVars\\
MultiTranscriptVar & $\rightarrow$ & Ref `[` ExtendedRawVar (`;' ExtendedRawVar)$^*$ (`,' ExtendedRawVar (`;' ExtendedRawVar)$^*$)$^+$ `]'\\
UnkEffectVar & $\rightarrow$ & Ref (`(=)' $|$ `?')\\
SplicingVar & $\rightarrow$ & Ref (`spl?' $|$ `(spl?)')\\
NoRNAVar & $\rightarrow$ & Ref `0' `?'?\\
Var & $\rightarrow$ & SingleVar $|$ MultiVar $|$ MultiTranscriptVar $|$ UnkEffectVar $|$ NoRNAVar $|$ SplicingVar\\
Nest & $\rightarrow$ & `\{' SimpleAlleleVarSet `\}'
\end{longtable}


\newpage

\bibliography{/home/jfjlaros/projects/bibliography}{}
\bibliographystyle{plain}

\end{document}
